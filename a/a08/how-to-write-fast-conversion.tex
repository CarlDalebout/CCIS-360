\textsc{How to write fast conversions}

\textsc{Problem:} Convert $1011110_{2}$ to base 4.
\\
\textsc{Answer:}\vspace{-2mm}
\begin{answerlong}
\begin{align*}
1011110_{2} &= (01_2|01_2|11_2|10_2)_{4} \\
           &= (1_4|1_4|3_4|2_4)_{4} \\
           &= 1132_{4} 
\end{align*}
Therefore $1011110_{2} = 1132_{4}$.
\end{answerlong}

In general when doing a sequence of aligned computations, you do this
\begin{align*}
a + b + c &= d \times e \cdot f \\
          &= g_{h} + i \\ 
          &= j + k^{l} 
\end{align*}
Notice that the \verb!=! symbols are aligned.
Take a look at the \LaTeX\ code:
\begin{console}[fontsize=\footnotesize]
\begin{align*}
a + b + c &= d \times e \cdot f \\
          &= g_{h} + i \\ 
          &= j + k^{l} 
\end{align*}
\end{console}
The \verb!&! is the alignment character and the \verb!\\! is newline.  
The aligned computations must be enclosed in \verb!\begin{align*}! and
\verb!\end{align*}!.
