\makeatletter
\DeclareOldFontCommand{\rm}{\normalfont\rmfamily}{\mathrm}
\DeclareOldFontCommand{\sf}{\normalfont\sffamily}{\mathsf}
\DeclareOldFontCommand{\tt}{\normalfont\ttfamily}{\mathtt}
\DeclareOldFontCommand{\bf}{\normalfont\bfseries}{\mathbf}
\DeclareOldFontCommand{\it}{\normalfont\itshape}{\mathit}
\DeclareOldFontCommand{\sl}{\normalfont\slshape}{\@nomath\sl}
\DeclareOldFontCommand{\sc}{\normalfont\scshape}{\@nomath\sc}
\makeatother

\input{myquizpreamble}
\input{yliow}
\input{ciss360}
\textwidth=6in

\renewcommand\TITLE{Final f01}

\newcommand\topmattertwo{
\topmatter
\score \\ \\
Open \texttt{main.tex} and enter answers (look for
\texttt{answercode}, \texttt{answerbox}, \texttt{answerlong}).
Turn the page for detailed instructions.
To rebuild and view pdf, in bash shell execute \texttt{make}.
To build a gzip-tar file, in bash shell execute \texttt{make s} and
you'll get \texttt{submit.tar.gz}.
}

\newcommand\tf{T or F or M}
\newcommand\answerbox[1]{\textbox{#1}}
\newcommand\codebox[1]{\begin{console}#1\end{console}}

\usepackage{pifont}
\newcommand{\cmark}{\textred{\ding{51}}}
\newcommand{\xmark}{\textred{\ding{55}}}

\newcounter{qc}
\newcommand\nextq{
%\newpage
\addtocounter{qc}{1}
Q{\theqc}.
}

\DefineVerbatimEnvironment%
 {answercode}{Verbatim}
 {frame=single,fontsize=\footnotesize}

\newenvironment{largebox}[1]{%
 \boxparone{#1}
}
{}

\usepackage{environ}
\let\oldquote=\quote
\let\endoldquote=\endquote
\let\quote\relax
\let\endquote\relax

% ADDED 2021/09/09
\renewcommand\boxpar[1]{
 \[
  \framebox[\textwidth][c] {
   \parbox[]{\dimexpr\textwidth - 0.25cm} {#1}
  }
 \]
}

\NewEnviron{answerlong}%
  {\global\let\tmp\BODY\aftergroup\doboxpar}

\newcommand\doboxpar{%
  \let\quote=\oldquote
  \let\endquote=\endoldquote
  \boxpar{\tmp}
}

\newenvironment{mcq}[7]%
{% begin code
#1 \dotfill{#2}
 \begin{tightlist}
 \item[(A)] #3
 \item[(B)] #4
 \item[(C)] #5
 \item[(D)] #6
 \item[(E)] #7 
 \end{tightlist}
}%
{% end code
} 

\renewcommand\EMAIL{}
\newcommand\POINTS{\textwhite{xxx/xxx}}
\newcommand\score{%
\vspace{-0.6in}
\begin{flushright}
Score: \answerbox{\POINTS}
\end{flushright}
\vspace{-0.4in}
\hspace{0.7in}\AUTHOR
\vspace{0.2in}
}

\newcommand\blankline{\mbox{}\\ }


\renewcommand\AUTHOR{jdoe5@cougars.ccis.edu} % CHANGE TO YOURS

\begin{document}
\topmattertwo

FALL2023 NEW INSTRUCTIONS:
The first bullet point below is changed to the following --
This is open-book and takehome and no discussion with
anyone (but yourself).
You can only use my notes.
Sure you can use a calculator, but you have to justify all calculations
anyway.

\textsc{Instructions}
\begin{enumerate}
\li This is a closed-book, no-discussion, no-calculator, no-browsing-on-the-web
    no-compiler/no-MIPS-simulator test.
\li Cheating is a serious academic offense. If caught you will 
    receive an immediate score of -100\%.
\li If a question asks for a program output and the program or
    code fragment contains
    an error, write \verb!ERROR! as output.
    When writing output, whitespace is significant.
\li If a question asks for the computation of a value and the program or
    code fragment contains
    an error, write \verb!ERROR! as value.
\end{enumerate}

\vspace{1cm}



\begin{center}
  \textsc{Honor Statement}
\end{center}
I, \answerbox{[REPLACE WITH YOUR FULLNAME]},
attest to the fact that the submitted work is my own and
is not the result of plagiarism.
Furthermore, I have not aided another student in the act of
plagiarism.

% ------------------------------------------------------------------------------

\newpage
\texttt{IMPORTANT INSTRUCTION FROM EARTH:
\\
SHOW AND EXPLAIN ***ALL*** YOUR WORK.
\\
YOU ARE NOT ALLOWED TO SHARE THIS DOCUMENT AFTER THE EXAM.
\\
END OF TRANSMISSION FROM HOUSON.
}

% ------------------------------------------------------------------------------
\begin{python}
from scoretable import *
\end{python}

% ------------------------------------------------------------------------------
\newpage
Welcome to your first day of work at Fullabugz Company.
Your job is to write MIPS
code, debug MIPS code (written by someone else ... ARGHHH), etc.

%By the way, your pay
is in jeopardy if the boss can't read your handwriting.
So write neatly!

Specifically:
\begin{itemize}
\li Write your scratch work on scratch paper. 
\li To correct code, circle or box up and cross out the wrong code, write and circle or box up the correct code somewhere, and draw an arrow from your correct code to the code that is crossed out.
\end{itemize}
Here are some examples:
\begin{python}
from latextool_basic import *
p = Plot()
p += ellipse(x0=0,y0=0,x1=5,y1=2,label=r'\texttt{add s0,s0,s0}',name='A')
p += cross(x0=0,y0=0,x1=5,y1=2)

p += ellipse(x0=7,y0=0+1,x1=7+5,y1=2+1,label=r'\texttt{add \$s0,\$s0,\$s0}',name='B')
p += Line(names=['B', 'A'], endstyle='>')
print(p)
\end{python}
\begin{python}
from latextool_basic import *
p = Plot()
p += Rect(x0=0,y0=0,x1=5,y1=2,label=r'\texttt{add s0,s0,s0}',name='A')
p += cross(x0=0,y0=0,x1=5,y1=2)

p += Rect(x0=7,y0=0+1,x1=7+5,y1=2+1,label=r'\texttt{add \$s0,\$s0,\$s0}',name='B')
p += Line(names=['B', 'A'], endstyle='>')
print(p)
\end{python}


Do not give me two answers for the same question.
I reserve the right to choose one
for you ...  and I always choose the wrong one (if there is one)!

Here's some useful info:
\begin{enumerate}[nosep]
\item Print int: syscall 1
\item Read int: syscall 5
\item Print string: syscall 4
\item Read string: syscall 8
\item Exit: syscall 10
\end{enumerate}
You should know what to do with \verb!$v0!, \verb!$a0!, \verb!$a1!, right?
\vspace{1cm}


% ------------------------------------------------------------------------------
\newpage
\nextq
The saga of an interplanetary software consultant continues ...

The fee of your consulting work for planet Tutu is \$51.765.
But they only accept your bill in base 2.
Convert 51.765 to base 2 up to 3 binary place (but with possible rounding
from the 4th binary place.) Show all work.

\textsc{Answer:}
In the following use \verb!_! for subscript and \verb!^! for power.

\begin{answercode}
First we consider 51.

    2 | 51
      +-------
    2 | 25 r 1
      +-------
    2 | 12 r 1
      +-------
    2 |  6 r 0
      +-------
    2 |  3 r 0
      +-------  
    2 |  1 r 1
      +-------
    2 |  0 r 1
      +-------
      
Therefore 51_{10} = 110011_2

Now we consider 0.765_{10}.

      2 x 0.765 = 1.53 r 1
      2 x 0.53  = 1.06 r 1
      2 x 0.06  = 0.12 r 0
      2 x 0.12  = 0.24 r 0

Therefore 0.765_{10} is approxmately 0.1100_{2} up to 4 binary place.
Therefore up to 3 binary places (with rounding), 0.765_{10} = 0.110_{2}.

Therefore 51.765_{10} = 110011.110_{2}.
\end{answercode}

% ------------------------------------------------------------------------------
\newpage
\nextq
Something seems to wrong with the mainframe computer on planet Rouf.
Their machine is basically in base 4.
After tracing some output, you suspect something is wrong with the base 4
multiplication in the chip.
You decide to double check at least one case by hand ...
\begin{enumerate}
\item[(a)] Compute the following product in base 4:
\[
120021_4 \times  103011_4
\]
\item[(b)] Convert the result of (a) to base 10.
\item[(c)] Convert $120021_4$ to base 2 quickly.
\item[(d)] Convert $103011_4$ to base 2 quickly.
\end{enumerate}

\textsc{Answer:}
In the following use \verb!_! for subscript and \verb!^! for power.


\begin{answercode}
(a)
                    120021
                  x 103011
           ---------------
                  111
                   +120021
                  +1200210
                       +00
               +1020123000
                     +0000
              +12002100000
           ---------------
               13030203231
           ---------------

(b) 13030203231_4 = 0001 1100 1100 1000 1110 1101 (complete this expression using ^ for power)
            = 1x10^6 + 8x10^5 + 8x10^4 + 6x10^3 + 4x10^2 + 4x10^1 + 5x10^0
            = 1886445

(c) 120021_4 = (1|2|0|0|2|1)_4
             = (01|10|00|00|10|01)_2
             = 0110 0000 1001_2

(d) 103011_4 = 0100 1100 0101_2

\end{answercode}

% ------------------------------------------------------------------------------
\newpage
\nextq
While traveling back to Earth, you were abducted by the Glinkons:
unfortunately your teleporter broke down and in an attempt to get home
in time for the beginning of the football season,
you thought it was safe to take a shortcut through their spacial territory ...

When their commander accused of spying on their planet,
you blurted out that you're just a software engineer ... a geek ...

\lq\lq Oh a CS geek, eh? Really?!?
In that case you won't have trouble solving this problem ..."

\lq\lq The number in 1011001 in base 2 has three 0s  and a total length of 7.
So the ratio of 0s is 3/7.
Right? Of course left trailing zeroes do not count."

(You're sweating now ...
it's one of those pesky number base problems again ... )

\lq\lq But this same number 1011001 in base 2 when converted to 10 is 89 which
has no zeros. So the ratio of 0s is 0/2."
 
\lq\lq So I'm giving you the number 49618850 in base 10.
Find the base B such that when this number is converted to this base B, the
representation has the largest ratio of zeros. B is at most 16."

\lq\lq You have 1 hour. SHOW ALL YOUR WORK!!!"

\textsc{Answer:}
\begin{answercode}
base B    |     representation in base B     | length  | number of zeroes | ratio
------    +----------------------------------+---------+------------------+------
     2    |0010 1111 0101 0001 1111 1010 0010|       26|                10|26/10
     3    |          0001 0110 1002 2002 1012|       17|                 7| 7/17
     4    |              02 33 11 01 33 22 02|       13|                 2| 2/13
    10    |                          49618850|        8|                 1| 1/8
    16    |                         02F5 1FA2|        7|                 0| 0/7

Therefore the base with the largest ratio of zeroes is B = 2.

\end{answercode}

% ------------------------------------------------------------------------------
\newpage
\nextq
(a) Many of the computers at planet Ducloz Dusun are all fried because
of solar flares ... and you're on that planet right now to help them.
Fortunately many computers shutdown in time.
To bring up all the functioning hardware, you need to key in the sys admin
password, which is the binary sequence of the two's complement representation
of $-96$ on their mainframe.
(They can't do it because they have forgotten their binary math.)
Note that their word size is $35$ bits, i.e. registers are made up of $35$ bits.

(b) The people Ducloz Dusun felt dumb,
not able to restart their machines because they forgot their binary math for
$5000$ years.
They wanted to learn their math all over again and your boss happily obliged
because he's going to charge them 5x the usual overtime pay.
Unfortunately you have to teach them the computer math.
In particular, you have to show them how to perform
\[
42 - 13
\]
on their 35 bit machine but using only addition.

\textsc{Answer:}
\begin{answercode}
Write down 42 and -13 in base 2 for 35-bit wide registers.
Perform 35-bit wide addition. Convert that back to base 10.

    42 =  000 0000 0000 0000 0000 0000 0000 0010 1010_2

    13 =  000 0000 0000 0000 0000 0000 0000 0000 1101_2

Flipping the bits of 13, we get:

    111 1111 1111 1111 1111 1111 1111 1111 0010_2

Adding a 1, we get -13:

    111 1111 1111 1111 1111 1111 1111 1111 0011_2

Therefore 42 - 13 = 42 + (-32) is

       000 0000 0000 0000 0000 0000 0000 0010 1010_2
	+  111 1111 1111 1111 1111 1111 1111 1111 0011_2
	-------------------------------------------------
	   000 0000 0000 0000 0000 0000 0000 0001 1101_2
	-------------------------------------------------

\end{answercode}

% ------------------------------------------------------------------------------
\newpage
\nextq
Great ... your boss bought a nanophone, a really tiny mobile phone.
MIPS is already a RISC machine, but your manager's nanophone is a reduced MIPS.
(RRISC?)
It particular, it does not have arithmetic instructions like add, addi, addiu,
sub, etc. (This is definitely not a smartphone ...)

Your manager asked you to write a function to perform unsigned addition of two
registers.
You checked the technical documentation for this reduced MIPS chip and see that
although it does not have arithmetic instructions, it does have all other
instructions and pseudoinstructions.
It also contains logical operations (i.e. bit operations). For instance:
\begin{Verbatim}[frame=single,fontsize=\small]
                      # Description              Description using C/C++
sll   $s0, $s1, 2     # shift left by constant   s0 = s1 << 2	
srl   $s0, $s1, 3     # shift right by constant  s0 = s1 >> 3
sllv  $s0, $s1, $t0   # shift left by variable   s0 = s1 << t0
srlv  $s0, $s1, $t1   # shift right by variable  s0 = s1 >> t1
                                                 
and   $s0, $s1, $s2   # and                      s0 = s1 & s2
andi  $s0, $s1, 4     # and immediate            s0 = s1 & 5
                                                 
or    $s0, $s1, $s2   # or                       s0 = s1 | s2
ori   $s0, $s1, 5     # or immediate             s0 = s1 | 5
                                                 
nor   $s0, $s1, $s2   # not-or (nor)             s0 = ~(s1 | s2)
\end{Verbatim}
(See chapter \#25 of my notes.)
You should call your function \verb!myaddu!.
Of course it performs unsigned add on \verb!$a0!, \verb!$a1! and puts the
resulting bits in \verb!$v0!.
You should add test code as given below.

Your manager is waiting for your work and will be testing it on PCSpim before
using the code in his nanophone.
He's also going to get Joe Cantcode to visually verify that your function does
NOT contain any arithmetic instructions (i.e., no addition, no subtraction,
no multiplication, no division, whether signed or unsigned.)

\textsc{Answer:}
Copy and paste your code below. I must be able to run it.
\begin{answercode}
        .text
        .globl main



myaddu: 
        # Perform unsigned add on $a0, $a1 and store result in $v03
        # get the last bit for a0 and a1 and conduct a half adder on them storing the carry
        li  $t1, 0  # s
        li  $t2, 0  # c
        li  $v0, 0  # ret
        li  $t0, 31 # iterator
        myaddu_loop:
            srlv    $t1, $a0, $t0
            andi    $t3, $t1, 1 # t3 = x_bit
            srlv    $t1, $a1, $t0
            andi    $t4, $t1, 1 # t4 = y_bit

            xor     $t5, $t3, $t4
            xor     $t1, $t5, $t2 # t1 = s
            
            and     $t5, $t5, $t2
            and     $t2, $t3, $t4
            or      $t2, $t5, $t2 # t2 = c

            sllv    $t5, $t1, $t0
            or      $v0, $v0, $t5

            s1l     $a0 $a0, 1
            s1l     $a1 $a1, 1
        beq $a0, $0, myaddu_loop_exit:
            j myaddu
        myaddu_loop_exit:
        jr $ra


main:                           # perform syscall to read int x and y and
                                # store them in $a0, $a1
        li      $v0 5
        syscall
        move    $a0, $v0
        
        li      $v0 5
        syscall
        move    $a1, $v0

        jal     myaddu          # compute the unsigned sum of $a0, $a1
                                # and store result in $v0
        move    $a0, $v0        # perform syscall to print int the return
        li      $v0, 1          # value in $v0 which is set by myaddu
        syscall

        li      $v0, 10
        syscall
\end{answercode}

%------------------------------------------------------------------------------
\newpage
\nextq
Great. Just great. Someone at Fullabugz accidentally deleted the source code
for one of their most important program ... the payroll program!!!
Everyone is given some MIPS machine code to disassemble, i.e.,
to translate into MIPS assembly code.
Here's yours machine code in hexadecimal:
\[
\texttt{0x01099022}
\]
Your boss told you that the above machine code is an R-format instruction.
You quickly google for MIPS machine instruction format and quickly look for
R-format instructions ...

(a) First the manual tells you to translate the above instruction
\verb!0x01099022! in hexadecimal into 32 bits.

(b) The manual tells you that the 32 bits from (a) is broken up into fields.
Going left to right, the first 6 bits is called \verb!op!,
the next 5 bits is called \lq\lq rs", and next 5 bits is called
\lq\lq rt", the next 5 bits is called \lq\lq rd",
the next 5 bits is called \lq\lq shamt", and the remaining rightmost 6 bits is
called \lq\lq funct". Write down the fields in base 2.

(c) In base 10, write down the op, shamt, and funct for the machine
instruction \verb!0x01099022!.

(d) The manual tells that rs, rt, rd are translated to registers by translating
the base 2 numbers rs, rt, rd to base 10.
For instance if the binary value of rt is 8, then it refers to register
\verb!$8!, i.e., it refers to the register name \verb!$t0!.
Write down the registers rs, rt, rd in base 10 and then their names.

(e) Finally, the manual tells us that for an R-format machine instruction is translated to the following assembly instruction:
\begin{Verbatim}
    [operation]    [rd], [rs], [rt]
\end{Verbatim}
\textsc{Answer:}
\begin{answercode}
(a) The machine code in 32 bits (for readability, group bits in chunks of 4
bits):

	0x01099022 = 0000 0001 0000 1001 1001 0000 0010 0010

(b) We break up the bits from (a) into chunks of 6 bits, 5 bits, 5 bits,
5 bits, 5 bits, 6 bits to get the op, rs, rt, rd, shamt, funct fields:

    000000 01000 01001 10010 00000 100010

The fields of the machine code 0x01099022 is

                 in base 2    in base 10
        op       00000	      0			
        rs       01000        8			
        rt       01001        9			
        rd       10010        18			
        shamt    00000        0			
        funct    100010       34			

(c) The operation is sub.

(d) For our machine instruction the register names of the rs, rt, rd are:

                register name
        rs      t0
        rt      t1
        rd      s2

(e) Finally the assembly instruction for the machine code 0x01099022 is

        sub $s2, $t0, $t1
\end{answercode}

% ------------------------------------------------------------------------------
\newpage
\nextq
The CEO has decreed that your pay is going to be \verb!f(10)! where
\verb!f! is the function:
\begin{Verbatim}[frame=single,fontsize=\small]
int f(int n)
{
    if (n == 0) return 5;
    else if (n == 1) return 7;
    else if (n == 2) return 3;
    else return 2 * f(n - 1) + 3 * f(n - 2) + 4 * f(n - 3);
}
\end{Verbatim}
but only if you can implement the above in MIPS.
What kind of a crazy company is this?!?

\textsc{Answer:}
Copy and paste your code below. I must be able to run it as a complete program.
\begin{answercode}
        .text
        .globl main

f:      # TODO
    beq     $a0, $0, f_return5
    
    li      $t0, 1    
    beq     $a0, $t0, f_return7
    
    li      $t0, 2
    beq     $a0, $t0, f_return3
    
    addiu   $sp, $sp, -8
        sw      $a0, 0($sp)
        sw      $ra, 4($sp)
        
            addi    $a0, $a0, -1
            jal     f
            move    $t0, $v0
            
            lw      $a0, 0($sp)
            addi    $a0, $a0, -2
            jal     f
            move    $t1, $v0

            lw      $a0, 0($sp)
            addi    $a0, $a0, -3
            jal     f
            move    $t2, $v0

            li      $t3, 2
            mult    $t0, $t3
            mflo    $v0

            li      $t3, 3
            mult    $t1, $t3
            mflo    $t0
            add     $v0, $v0, $t0

            li      $t3, 4
            mult    $t2, $t3
            mflo    $t0
            add     $v0, $v0, $t0
            
        lw      $ra, 4($sp)
        lw      $a0, 0($sp)
    addiu   $sp, $sp,  8
    jr      $ra

    f_return5:
        li      $v0, 5
        jr      $ra
    f_return7:
        li      $v0, 7
        jr      $ra
    f_return3:
        li      $v0, 3
        jr      $ra
    

main:                       # perform syscall to read int n and
                            # store them in $a0
    li      $v0, 5
    syscall
    move    $a0, $v0
    
    jal     f               # compute f(a0) and store result in $v0
    move    $a0, $v0
    
    li      $v0, 1          # perform syscall to print int the return
    syscall                 # value in $v0 which is set by f

    li      $v0, 10
    syscall
\end{answercode}
Note:
Of course you are advised to test your code.
As always, if your program crashes or runs an infinite loop during a test run,
you get 0 for that test case.

%------------------------------------------------------------------------------
\newpage
\nextq
Your work on Planet Taolf, their civilization now has advanced to the point
where they now need more than just integers!
Time for floating point numbers!
This curious civilization manage to move to computers with only the concept
of integers.
You have decided not to contact IEEE to design a new floating point type --
you are going to design it yourself (because you don’t want to outsource and
you want to keep all the consulting fees).
Well, not exactly, you are going to copy their design.
There’s just one problem:
their computers uses 30 bits registers and not 32.
You have decided to follow the IEEE 754 Single Precision representation as
much as possible.
The 30-bit computers on Toalf will be laid out with 3 fields just like IEEE 754
Single Precision format:
\begin{enumerate}[nosep]
\li 1 bit for the sign bit s.
\li 8 bits for encoding the exponent E. The bias is the same, i.e., 127.
\li 21 bits for the fractional part F of the mantissa.
\end{enumerate}
You are calling this the TAOLF30 standard.

\begin{enumerate}
\item[(a)] You are going to test your Toalf computer with TOALF30 floating
point.
Here’s your test case: What is the 30-bit number
\verb!011011010010101011111000001001! written in TAOLF30 in our usual
normalized base 10 scientific notation?
\item[(b)] Now you need to bill Toalf.
It’s going to be \$57,130,500.55.
But you are now on Earth.
So you need to convert 57130500.55 into 30 bits and send the bits to
Toalf by intergalactic-network. What is the 30-bit representation of
57130500.55 using the TAOLF30 representation?
\end{enumerate}

\textsc{Answer:}
\begin{answercode}
(a) The fields of 0 11011010 010101011111000001001 are

    s = 0
    E + Bias =  218
    F = 0.704009

Therefore

    sign = + (put + or -)
    E = 91
    1 + F = 1010101011111000001001

Converting to base 10

    sign = + (put + or -)
    E = 91
    1 + F = 2.801161

Hence in signed fractional binary form, 0 11011010 010101011111000001001 is

    (+ or -) 1.010101011111000001001 x 2^91 =  
    
Converting the above to normalized base 10 scientific notation, we get

1.335697650909423828125 x 2^91 = 3,307,027,204,880,404,564,110,475,264

(b)
\end{answercode}


%------------------------------------------------------------------------------
\newpage
\nextq
You are told to design the following for a game show on planet Nemow.
The design will tell the game host who which of two teams will get to answer
the question.

There are two husband-and-wife teams.
Let's call the input from the husband of the first team $w$ and the input
his wife $x$.
The input from the husband of the second is $y$ and the input from his wife
is $z$.
An output of 0 mean team 1 will answer and an output of 1 will mean team 2
gets to answer.

On planet Nemow, during election, each woman's vote count as 2 points and a
man's vote count as 1.
The game show host has decided to use the same rule for his show.
After reading a question, members can press a button to indicate interest in
answering the question.
If the husband from team 1 presses his button, which for us will mean $w = 1$,
and his wife also presses her button, i.e., $x = 1$, they get 3 points.
Suppose for team 2, the husband does not press his button, i.e., y = 0,
but his wife presses her button, i.e., $z = 1$.
So team 2 gets 2 points.
Your machine will decide based on which team has more \lq\lq button points".
In this case, since team 1 has 3 points and team 2 has 2 points,
team 1 gets to answer.
Your machine will output $f = 0$.
An output of $f = 1$ means team 2 gets to answer.
For instance suppose for team 1, the husband presses the button and his wife
does not (team 1 has 1 button point), and for team 2, the husband does not
press his button but his wife does (team 2 has 2 button points).
Therefore in this case team 2 gets to answer; your device will output $f = 1$.

In the case of a tie breaker, team 1 gets to answer.
That sounds like a good deal.
However, if no one wants to answer, team 1 has to answer!

Fortunately you only need to provide a logic design.
Make sure you work out each part carefully and correctly.
I will stop reading your work once you hit a mistake.

\begin{enumerate}[nosep]
\item[(a)] Write down the truth table for $f$.
\item[(b)] Write down the K-map for $f$.
\item[(c)] Derive a minimal boolean expression for $f$ (in SOP form).
\item[(d)] Draw the logic network diagram for $f$. (See below for example.)
\item[(e)] Write down the cost of the logic network.
(This can be calculated this way: Ignoring the NOT gates,
count the number of inputs to all the AND and the OR gates.)
\end{enumerate}

\textsc{Answer:}
\begin{answercode}

(a)      +---------+---+
         | w x y z | f | 
         +---------+---+
         | 0 0 0 0 | 0 |
         | 0 0 0 1 | 1 |
         | 0 0 1 0 | 1 |
         | 0 0 1 1 | 1 |
         | 0 1 0 0 | 0 |
         | 0 1 0 1 | 0 |
         | 0 1 1 0 | 0 |
         | 0 1 1 1 | 1 |
         | 1 0 0 0 | 0 |
         | 1 0 0 1 | 1 |
         | 1 0 1 0 | 0 |
         | 1 0 1 1 | 1 |
         | 1 1 0 0 | 0 |
         | 1 1 0 1 | 0 |
         | 1 1 1 0 | 0 |
         | 1 1 1 1 | 0 |
         +---------+---+
         
(b)   yz   00  01  11  10
   wx    +----------------+
      00 |  0   1   1   1 |
      01 |  0   0   1   0 |
      11 |  0   0   0   0 |
      10 |  0   1   1   0 |
         +----------------+

(c) f = wx'z + w'x'z + w'yz + w'x'y

(d) 
                             +---+ wx'z
    w---+--------------------|AND|------------+                 
        |       +------------|   |            |           
        |       |  +---------|   |            |     
        |       |  |         +---+            |     
        |       |  |           +---+ w'x'z    |     
        +--NOT--|--|--+--------|AND|-------+  |           
                |  |  |  +-----|   |       |  |           
    x---+-------+  |  |  |  +--|   |       |  |           
        |          |  |  |  |  +---+       |  |  +---+ wx'z + w'x'z + w'yz + w'x'y
        +--NOT-----|--|--+  |              |  +--|OR |---------------------------------
                   |  |  |  |              +-----|   |     
                   |  |  |  |              +-----|   |     
                   |  |  |  |              |  +--|   |     
                   |  |  |  |  +---+ w'yz  |  |  +---+           
                   |  +--|--|--|AND|-------+  |             
    y-------+------|--|--|--|--|   |          |         
            |      +--|--|--+--|   |          |         
            |      |  |  |     +---+          |             
    z-------|------+  |  |                    |         
            |         |  |  +---+ w'x'y       |         
            +---------|--|--|AND|-------------+       
                      +--|--|   |                           
                         +--|   |                           
                            +---+                       




(e) Cost = 4 + 4 + 4 + 4 + 5 = 21
\end{answercode}
Here's an example of how to draw logic networks using ASCII art:
\begin{Verbatim}[frame=single]
                          +---+ ab
      a ---+--------------|AND|-----+
           |          +---|   |     |   +---+ ab + a'b
           +--NOT--+  |   +---+     +---|OR |----------
                   |  |             +---|   |
                   |  |   +---+     |   +---+
                   +--|---|AND|-----+
      b --------------+---|   | a'b
                          +---+
\end{Verbatim}
The above logic network has a cost of 2 + 2 + 2 = 6.

%------------------------------------------------------------------------------
\newpage

\textsc{Instructions}

In \verb!main.tex! change the email address in
\begin{console}
\renewcommand\AUTHOR{jdoe5@cougars.ccis.edu} 
\end{console}
yours.
In the bash shell, execute \lq\lq \verb!make!" to recompile \verb!main.pdf!.
Execute \lq\lq \verb!make v!" to view \verb!main.pdf!.
Execute \lq\lq \verb!make s!" to create \verb!submit.tar.gz! for submission.

For each question, you'll see boxes for you to fill.
You write your answers in \verb!main.tex! file.
For small boxes, if you see
\begin{console}[frame=single=single,fontsize=\small]
1 + 1 = \answerbox{}.
\end{console}
you do this:
\begin{console}[frame=single=single,fontsize=\small]
1 + 1 = \answerbox{2}.
\end{console}
\verb!answerbox! will also appear in
\lq\lq true/false" and \lq\lq multiple-choice"
questions.

For longer answers that needs typewriter font, if you see
\begin{console}[frame=single=single, fontsize=\small]
Write a C++ statement that declares an integer variable name x.
\begin{answercode}
\end{answercode}
\end{console}
you do this:
\begin{console}[frame=single=single, fontsize=\small]
Write a C++ statement that declares an integer variable name x.
\begin{answercode}
int x;
\end{answercode}
\end{console}
\verb!answercode! will appear in questions asking for
code, algorithm, and program output.
In this case, indentation and spacing is significant.
For program output, I do look at spaces and newlines.

For long answers (not in typewriter font) if you see
\begin{console}[frame=single=single, fontsize=\small]
What is the color of the sky?
\begin{answerlong}
\end{answerlong}
\end{console}
you can write
\begin{console}[frame=single=single, fontsize=\small]
What is the color of the sky?
\begin{answerlong}
The color of the sky is blue.
\end{answerlong}
\end{console}
For students beyond 245: You can put \LaTeX\ commands in \verb!answerlong!.

A question that begins with \lq\lq T or F or M"
requires you to identify whether it is true or
false, or meaningless.
\lq\lq Meaningless" means something's wrong with the statement and
it is not well-defined.
Something like \lq\lq $1 +_2$" or \lq\lq $\{2\}^{\{3\}}$" is not
well-defined.
Therefore a question such as
\lq\lq Is $42 = 1 +_2$ true or false?" or
\lq\lq Is $42 = \{2\}^{\{3\}}$ true or false?"
does not make sense.
\lq\lq Is $P(42) = \{42\}$ true or false?" is meaningless because $P(X)$
is only defined if $X$ is a set.
For \lq\lq Is 1 + 2 + 3 true or false?", \lq\lq 1 + 2 + 3" is well--defined but
as a
\lq\lq numerical expression", not as a \lq\lq proposition", i.e.,
it cannot be true or false.
Therefore \lq\lq Is 1 + 2 + 3 true or false?" is also not a well-defined
question.

When writing results of computations, make sure it's simplified.
For instance write $2$ instead of $1 + 1$.
When you write down sets,
if the answer is $\{1\}$, I do not
want to see $\{1, 1\}$.

When writing a counterexample, always write the simplest.

Here are some examples (see \verb!instructions.tex! for details):

\begin{enumerate}

 \item \tf: 1 + 1 = 2 \dotfill\answerbox{T}
 
 \item \tf: 1 + 1 = 3 \dotfill\answerbox{F}
 
 \item \tf: $1 +^2 =$ \dotfill\answerbox{M}
 
 \item $1 + 2 =$ \answerbox{3}
 
 \item Write a C++ statement to declare an integer variable named
 \verb!x!.
 \begin{answercode}
int x;
 \end{answercode}

 \item Solve $x^2 - 1 = 0$.
 \begin{answerlong}
 Since $x^2 - 1 = (x-1)(x+1)$, $x^2 - 1 = 0$ implies $(x-1)(x+1)=0$.
 Therefore $x - 1 = 0$ or $x = -1$.
 Hence $x = 1$ or $x = -1$.
 \end{answerlong}

 \item
 \begin{mcq}
 {Which is true?}{\answerbox{C}}
 {$1+1=0$}
 {$1+1=1$}
 {$1+1=2$}
 {$1+1=3$}
 {$1+1=4$}
 \end{mcq}


\end{enumerate}

\end{document}