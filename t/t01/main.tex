\makeatletter
\DeclareOldFontCommand{\rm}{\normalfont\rmfamily}{\mathrm}
\DeclareOldFontCommand{\sf}{\normalfont\sffamily}{\mathsf}
\DeclareOldFontCommand{\tt}{\normalfont\ttfamily}{\mathtt}
\DeclareOldFontCommand{\bf}{\normalfont\bfseries}{\mathbf}
\DeclareOldFontCommand{\it}{\normalfont\itshape}{\mathit}
\DeclareOldFontCommand{\sl}{\normalfont\slshape}{\@nomath\sl}
\DeclareOldFontCommand{\sc}{\normalfont\scshape}{\@nomath\sc}
\makeatother

\input{myquizpreamble}
\input{yliow}
\input{ciss360}
\textwidth=6in

\renewcommand\TITLE{Final f01}

\newcommand\topmattertwo{
\topmatter
\score \\ \\
Open \texttt{main.tex} and enter answers (look for
\texttt{answercode}, \texttt{answerbox}, \texttt{answerlong}).
Turn the page for detailed instructions.
To rebuild and view pdf, in bash shell execute \texttt{make}.
To build a gzip-tar file, in bash shell execute \texttt{make s} and
you'll get \texttt{submit.tar.gz}.
}

\newcommand\tf{T or F or M}
\newcommand\answerbox[1]{\textbox{#1}}
\newcommand\codebox[1]{\begin{console}#1\end{console}}

\usepackage{pifont}
\newcommand{\cmark}{\textred{\ding{51}}}
\newcommand{\xmark}{\textred{\ding{55}}}

\newcounter{qc}
\newcommand\nextq{
%\newpage
\addtocounter{qc}{1}
Q{\theqc}.
}

\DefineVerbatimEnvironment%
 {answercode}{Verbatim}
 {frame=single,fontsize=\footnotesize}

\newenvironment{largebox}[1]{%
 \boxparone{#1}
}
{}

\usepackage{environ}
\let\oldquote=\quote
\let\endoldquote=\endquote
\let\quote\relax
\let\endquote\relax

% ADDED 2021/09/09
\renewcommand\boxpar[1]{
 \[
  \framebox[\textwidth][c] {
   \parbox[]{\dimexpr\textwidth - 0.25cm} {#1}
  }
 \]
}

\NewEnviron{answerlong}%
  {\global\let\tmp\BODY\aftergroup\doboxpar}

\newcommand\doboxpar{%
  \let\quote=\oldquote
  \let\endquote=\endoldquote
  \boxpar{\tmp}
}

\newenvironment{mcq}[7]%
{% begin code
#1 \dotfill{#2}
 \begin{tightlist}
 \item[(A)] #3
 \item[(B)] #4
 \item[(C)] #5
 \item[(D)] #6
 \item[(E)] #7 
 \end{tightlist}
}%
{% end code
} 

\renewcommand\EMAIL{}
\newcommand\POINTS{\textwhite{xxx/xxx}}
\newcommand\score{%
\vspace{-0.6in}
\begin{flushright}
Score: \answerbox{\POINTS}
\end{flushright}
\vspace{-0.4in}
\hspace{0.7in}\AUTHOR
\vspace{0.2in}
}

\newcommand\blankline{\mbox{}\\ }


\renewcommand\AUTHOR{jdoe5@cougars.ccis.edu} % CHANGE TO YOURS

\begin{document}
\topmattertwo

\textsc{Instructions}
\begin{enumerate}
\li This is a closed-book, no-discussion, no-calculator, no-browsing-on-the-web
    no-compiler/no-MIPS-simulator test.
\li Cheating is a serious academic offense. If caught you will 
    receive an immediate score of -100\%.
\li If a question asks for a program output and the program or
    code fragment contains
    an error, write \verb!ERROR! as output.
    When writing output, whitespace is significant.
\li If a question asks for the computation of a value and the program or
    code fragment contains
    an error, write \verb!ERROR! as value.
\end{enumerate}

\vspace{1cm}


\begin{center}
  \textsc{Honor Statement}
\end{center}
I, \answerbox{[REPLACE WITH YOUR FULLNAME]},
attest to the fact that the submitted work is my own and
is not the result of plagiarism.
Furthermore, I have not aided another student in the act of
plagiarism.

% ------------------------------------------------------------------------------
\begin{python}
from scoretable import *
\end{python}

% ------------------------------------------------------------------------------
\newpage
Welcome to your first day of work at Fullabugz Company.
Your job is to write MIPS
code, debug MIPS code (written by someone else ... ARGHHH), etc.

%By the way, your pay
is in jeopardy if the boss can't read your handwriting.
So write neatly!

Specifically:
\begin{itemize}
\li Write your scratch work on scratch paper. 
\li To correct code, circle or box up and cross out the wrong code, write and circle or box up the correct code somewhere, and draw an arrow from your correct code to the code that is crossed out.
\end{itemize}
Here are some examples:
\begin{python}
from latextool_basic import *
p = Plot()
p += ellipse(x0=0,y0=0,x1=5,y1=2,label=r'\texttt{add s0,s0,s0}',name='A')
p += cross(x0=0,y0=0,x1=5,y1=2)

p += ellipse(x0=7,y0=0+1,x1=7+5,y1=2+1,label=r'\texttt{add \$s0,\$s0,\$s0}',name='B')
p += Line(names=['B', 'A'], endstyle='>')
print(p)
\end{python}
\begin{python}
from latextool_basic import *
p = Plot()
p += Rect(x0=0,y0=0,x1=5,y1=2,label=r'\texttt{add s0,s0,s0}',name='A')
p += cross(x0=0,y0=0,x1=5,y1=2)

p += Rect(x0=7,y0=0+1,x1=7+5,y1=2+1,label=r'\texttt{add \$s0,\$s0,\$s0}',name='B')
p += Line(names=['B', 'A'], endstyle='>')
print(p)
\end{python}


Do not give me two answers for the same question.
I reserve the right to choose one
for you ...  and I always choose the wrong one (if there is one)!

Here's some useful info:
\begin{enumerate}[nosep]
\item Print int: syscall 1
\item Read int: syscall 5
\item Print string: syscall 4
\item Read string: syscall 8
\item Exit: syscall 10
\end{enumerate}
You should know what to do with \verb!$v0!, \verb!$a0!, \verb!$a1!, right?
\vspace{1cm}


% ------------------------------------------------------------------------------
\newpage
\nextq
\begin{itemize}

\item[(a)]
  ISA stands for
  
  \ANSWER\vspace{4mm}
  \begin{answercode}

  \end{answercode}

\item[(b)]
  CISC stands for
 
  \ANSWER\vspace{4mm}
  \begin{answercode}

  \end{answercode}

\item[(c)]
  RISC stands for
  \\
  \ANSWER\vspace{4mm}
  \begin{answercode}

  \end{answercode}
  
\item[(c)]
  \tf.
  An assembly instruction translates to one machine instruction.
  \\
  \ANSWER\vspace{4mm}
  \begin{answercode}

  \end{answercode}
  
\item[(d)]
  The five parts of a computer are
  \\
  \ANSWER\vspace{4mm}
  \begin{answercode}

  \end{answercode}

  
\item[(e)]
  The CPU or processor is made up of the following two parts:
  \\
  \ANSWER\vspace{4mm}
  \begin{answercode}

  \end{answercode}
  
\item[(f)]
  Which of the following provides the fastest
  storage mechanism. Choose one.
  \begin{itemize}
  \item[a.] CPU Cache
  \item[b.] Register file
  \item[b.] Memory
  \item[c.] External storage (such as hard drive)
  \end{itemize}
  \ANSWER\vspace{4mm}
  \begin{answercode}

  \end{answercode}
\end{itemize}

% ------------------------------------------------------------------------------
\newpage
\nextq
Joe Cantcode wrote the following code to display the first
integer in the data segment 
to the
console.
But it doesn't work!!! Correct it.
In the code below, the first integer in the data segment is 2010.
But you cannot assume that it won't change in the future.
You must not change the data segment.

(From the management: This is a warm-up. Just insert one line.)

\textsc{Answer:}
\begin{answercode}  
	  .text
	  .globl    main
          
main:     la        $a0, odyssey
	  li        $v0, 1
	  syscall
	  li        $v0, 10
	  syscall

          .data
odyssey:  .word 2010
\end{answercode}

% ------------------------------------------------------------------------------
\newpage
\nextq
Tom Toofaz always jumps into his project assignments and hack out code ... and
ends up spending his night in the office.
He's having problem implementing the following requirement
in MIPS with less than four \verb!t!--registers.
But his manager, Sloe Arnn Steady, told him that four is \textit{way} too many.
Help Tom and he'll buy you lunch.

Write a code segment in MIPS to compute
\verb!9 * a - (b + c) + d! where
\verb!$s0!, \verb!$s1!, \verb!$s2!, and
\verb!$s3! have values of the C/C++ integer variables
\verb!a!, \verb!b!, \verb!c!, \verb!d! respective.
The result must be placed in \verb!$s4!.
Your code must be as efficient as possible and you must use your registers
efficiently too.
You may assume that the \verb!t!-registers are not used elsewhere, and so you
can use the \verb!t!--registers in your code.
No such guarantee for the \verb!s!-registers -- therefore
you cannot change the values in
\verb!$s0!, \verb!$s1!, \verb!$s2!, and \verb!$s3!.
You also cannot use the \verb!a!-- and \verb!v!--registers.
Do not use the \verb!mul! or \verb!mult! command.
Write down the number of \verb!t!-registers used.
\\
\ANSWER
\begin{answercode}

\end{answercode}
\vspace{-4mm}
(Code fragment, not a complete program.)

Number of \verb!t!--registers used:
\\
\ANSWER
\begin{answercode}

\end{answercode}



% ------------------------------------------------------------------------------
\newpage
\nextq
The following is an attempt by Sue Longlunch to scan an array of 
words and to add 10 to each word in that array. 
The array has size 20 and base address is in register 
\verb!$s0!.
Your manager is fuming mad because there are bugs in the code
(and Sue is still at lunch) and have asked you to
do a code review and 
writing in some comments.
You should write \lq\lq \verb!should be: xxxxx!" to replace
the code or
\lq\lq\verb!insert here: xxxxx!" to insert some code between two lines.

\begin{answercode}
                                   COMMENTS TO SUEa
        addi    $t0, $s0, 20       should be: ????

loop:   blt     $s0, $t1, exit:    
                                   insert here: ????
        move    $t1, 0($s0)        

        addi    $s0, $s0, 1        

        addi    $t1, $t1, 10       
                                     
        jr      loop                 
exit:
\end{answercode}

% ------------------------------------------------------------------------------
\newpage
\nextq
While working on an artificial neural network,
your senior teammate Arnie Ned needs you to write a MIPS code fragment to 
\lq\lq clamp" the integer value in \verb!$s0! at \verb!-5! and \verb!5!.
In other words
if \verb!$s0! is less than \verb!-5!,
\verb!$s0! is set to \verb!-5!.
If the value of \verb!$s0! is
greater than \verb!5!,
\verb!$s0! is set to \verb!5!.
Otherwise, the value in \verb!$s0! is unchanged.

\textsc{Answer:}
\begin{answercode}

\end{answercode}
\vspace{-4mm}
(Code fragment, not a complete program.)

%------------------------------------------------------------------------------
\newpage
\nextq
What is the MIPS code fragment for the C code
\[
  \verb!f[i + 2] = f[i + 1] + f[i];!
\]
where \verb!f! is an array of integers.
The base
address of \verb!f! is associated with \verb!$t0!, \verb!i!
is associated with \verb!$t1!.

\textsc{Answer:}
\begin{answercode}

\end{answercode}

% ------------------------------------------------------------------------------
\newpage
\nextq
Your project leader Lee Pointa
wants you to write some MIPS code to perform the first pass of bubblesort
(in ascending order)
on an
integer array,
which will be used in a bubblesort that Bubba Short will use.
Lee says that the base address of the array is stored in \verb!$a0!
and the number of values in the array is stored in \verb!$a1!.
Another thing: Lee tells you he prefers you do not use an index variable, but
rather, use a pointer to iterate over the array.
(Look at his lastname.)

\textsc{Answer:}
\begin{answercode}

\end{answercode}
\vspace{-4mm}
(Code fragment, not a complete program.)

%------------------------------------------------------------------------------
\newpage
\nextq
It's lunchtime ... the company quiz for today is below.
The winner is the first to answer all questions correctly.
Work fast: Lee Pointa is really good with these type of questions.

(a) What's in \verb!$s1!, \verb!$s2!, and \verb!$s3!
at the end of the code fragment (i.e., not complete program)?
(Write ERROR if the value is not an integer value.)
\begin{Verbatim}[frame=single,fontsize=\small]
        .text
        .globl main        
        [code omitted]
        la     $s0, L
        lw     $s1, 8($s0)
        lw     $s2, 16($s0)
        la     $s0, L1
        lw     $s3, -4($s0)
        [code omitted]
        
        .data
L:      .word 42 43 44 45
        .word 46 47 48 49
L1:     .word 50 51 52 53
        .word 54 55 56 57
\end{Verbatim}
\textsc{Answer:}
\\
\verb!$s1! \answerbox{?} \hspace{1cm}
\verb!$s2! \answerbox{?} \hspace{1cm}
\verb!$s3! \answerbox{?} \\

(b)
What is the value of \verb!$s2! at the end of the code fragment?
(Write ERROR if the value is not an integer value.)
\begin{Verbatim}[frame=single,fontsize=\small]
        .text
        .globl main        
        [code omitted]
        la     $s0, W
        lw     $s1, 4($s0)
        lw     $s2, 8($s1)
        [code omitted]
        
        .data
X:      .word 1 2 3 4
Y:      .word 5 6 7 8
Z:      .word 9 10 11 12
W:      .word X Y Z
\end{Verbatim}
\textsc{Answer:} \\
\verb!$s2! \answerbox{?}

%------------------------------------------------------------------------------
\newpage
\nextq
Write a MIPS code fragment that, starting with $n$ (assume stored in
\verb!$s0!),
it
continually computes $3n + 1$ if $n$ is odd or $n/2$
if $n$ is odd
and store it back into $n$ until $n$ is $1$.
(\verb!/! is integer division.)
The values of $n$ are stores in an array in the data segment --
a label \verb!L! is already created for you (you don't have to create it).
For instance if the user
enters $3$, then the values are
\[
  3, 10, 5, 16, 8, 4, 2, 1
\]
For instance when $n = 3$, since $n$ is odd,
the next value of $n$ is $3n + 1 = 10$.
The next value for $n = 10$, since $n = 10$ is even,
is $n / 2 = 10 / 2 = 5$.
The values
\[
  3, 10, 5, 16, 8, 4, 2, 1
\]
are stored at the beginning of the data
segment.
Note that the starting value $n = 3$ is also stored.

\textsc{Answer:}
\begin{answercode}

\end{answercode}

%------------------------------------------------------------------------------
\newpage
\nextq
Your boss just did a code review and found that Joe Cantcode
wrote some MIPS code implementing the following idea
in a critical part of the code to auto-correct
the tilt actuators on a high speed train:
\begin{Verbatim}[frame=single]
if (t0 == 0)
{
    t1 = 12;
}
else if (t0 == 1)
{
    t1 = 34;
}
else if (t0 == 2)
{
    t1 = 56;
}
else if (t0 == 3)
{
    t1 = 78;
}
\end{Verbatim}
where variables 
\verb!t0!,
\verb!t1!
and associated with registers
\verb!$t0!,
\verb!$t1!
respectively.
He has asked you rewrite the code to implement a switch 
in MIPS using the
jump table method 
\begin{Verbatim}[frame=single]
switch (t0)
{
    case 0:
        t1 = 12;
        break;
    case 1:
        t1 = 34;
        break;
    case 2:
        t1 = 56;
        break;
    case 3:
        t1 = 78;
}
\end{Verbatim}
so that he can do some 
performance analysis and see if the jump table method is faster.

\textsc{Answer.}
Implement the switch here
\begin{answercode}
        .text
        .globl main
main:
        li $t0, 0           # your code must work for different values of t0

        # implement switch below




        # At this point $t1 must be set to the right value based on your
        # implementation of switch
        li $v0, 10          # exit
        syscall

        .data

\end{answercode}

%------------------------------------------------------------------------------
\newpage
\nextq
Implement the following C/C++ code fragment in
MIPS code fragment
\[
  \verb!y2 = x * x * x + 2 * x * x + 3 * x + 4;!
\]
where \verb!y2! is \verb!$s0! and \verb!x! is \verb!$s1!.
Assume that the cost of
multiplication is higher than the cost of addition.
Say the cost of multiplication is 20 and addition is 1.
You want your MIPS code to be as fast as possible
since the above is used in performing 
error correction codes in signal processing.
The project leader, Zig Nell,
claims that it can be done with two multiplications.

\textsc{Answer:}
\begin{answercode}
  
\end{answercode}

% ------------------------------------------------------------------------------
\newpage
\nextq
Just great.
Bubba Short is not back from lunch.
Lee Pointa said you have to write the bubblesort as a function  for him.
Remember that
the base address of the array is stored in \verb!$a0!
and the number of values in the array is stored in \verb!$a1!.
Just to be somewhat organized Lee told you to include at least
a \verb!swap! function that swaps the integer values at
addresses stored in \verb!$a0! and \verb!$a1!.
(You can add more functions if you like.
For instance you might want to have a function that performs
one pass of bubblesort.)

\textsc{Answer:}
\begin{answercode}
swap:


bubblesort:


\end{answercode}

%------------------------------------------------------------------------------
\newpage

\textsc{Instructions}

In \verb!main.tex! change the email address in
\begin{console}
\renewcommand\AUTHOR{jdoe5@cougars.ccis.edu} 
\end{console}
yours.
In the bash shell, execute \lq\lq \verb!make!" to recompile \verb!main.pdf!.
Execute \lq\lq \verb!make v!" to view \verb!main.pdf!.
Execute \lq\lq \verb!make s!" to create \verb!submit.tar.gz! for submission.

For each question, you'll see boxes for you to fill.
You write your answers in \verb!main.tex! file.
For small boxes, if you see
\begin{console}[frame=single=single,fontsize=\small]
1 + 1 = \answerbox{}.
\end{console}
you do this:
\begin{console}[frame=single=single,fontsize=\small]
1 + 1 = \answerbox{2}.
\end{console}
\verb!answerbox! will also appear in
\lq\lq true/false" and \lq\lq multiple-choice"
questions.

For longer answers that needs typewriter font, if you see
\begin{console}[frame=single=single, fontsize=\small]
Write a C++ statement that declares an integer variable name x.
\begin{answercode}
\end{answercode}
\end{console}
you do this:
\begin{console}[frame=single=single, fontsize=\small]
Write a C++ statement that declares an integer variable name x.
\begin{answercode}
int x;
\end{answercode}
\end{console}
\verb!answercode! will appear in questions asking for
code, algorithm, and program output.
In this case, indentation and spacing is significant.
For program output, I do look at spaces and newlines.

For long answers (not in typewriter font) if you see
\begin{console}[frame=single=single, fontsize=\small]
What is the color of the sky?
\begin{answerlong}
\end{answerlong}
\end{console}
you can write
\begin{console}[frame=single=single, fontsize=\small]
What is the color of the sky?
\begin{answerlong}
The color of the sky is blue.
\end{answerlong}
\end{console}
For students beyond 245: You can put \LaTeX\ commands in \verb!answerlong!.

A question that begins with \lq\lq T or F or M"
requires you to identify whether it is true or
false, or meaningless.
\lq\lq Meaningless" means something's wrong with the statement and
it is not well-defined.
Something like \lq\lq $1 +_2$" or \lq\lq $\{2\}^{\{3\}}$" is not
well-defined.
Therefore a question such as
\lq\lq Is $42 = 1 +_2$ true or false?" or
\lq\lq Is $42 = \{2\}^{\{3\}}$ true or false?"
does not make sense.
\lq\lq Is $P(42) = \{42\}$ true or false?" is meaningless because $P(X)$
is only defined if $X$ is a set.
For \lq\lq Is 1 + 2 + 3 true or false?", \lq\lq 1 + 2 + 3" is well--defined but
as a
\lq\lq numerical expression", not as a \lq\lq proposition", i.e.,
it cannot be true or false.
Therefore \lq\lq Is 1 + 2 + 3 true or false?" is also not a well-defined
question.

When writing results of computations, make sure it's simplified.
For instance write $2$ instead of $1 + 1$.
When you write down sets,
if the answer is $\{1\}$, I do not
want to see $\{1, 1\}$.

When writing a counterexample, always write the simplest.

Here are some examples (see \verb!instructions.tex! for details):

\begin{enumerate}

 \item \tf: 1 + 1 = 2 \dotfill\answerbox{T}
 
 \item \tf: 1 + 1 = 3 \dotfill\answerbox{F}
 
 \item \tf: $1 +^2 =$ \dotfill\answerbox{M}
 
 \item $1 + 2 =$ \answerbox{3}
 
 \item Write a C++ statement to declare an integer variable named
 \verb!x!.
 \begin{answercode}
int x;
 \end{answercode}

 \item Solve $x^2 - 1 = 0$.
 \begin{answerlong}
 Since $x^2 - 1 = (x-1)(x+1)$, $x^2 - 1 = 0$ implies $(x-1)(x+1)=0$.
 Therefore $x - 1 = 0$ or $x = -1$.
 Hence $x = 1$ or $x = -1$.
 \end{answerlong}

 \item
 \begin{mcq}
 {Which is true?}{\answerbox{C}}
 {$1+1=0$}
 {$1+1=1$}
 {$1+1=2$}
 {$1+1=3$}
 {$1+1=4$}
 \end{mcq}


\end{enumerate}

\end{document}
